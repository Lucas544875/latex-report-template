\documentclass[uplatex,dvipdfmx,ja=standard,a4paper]{bxjsarticle}
%引用:https://qiita.com/rityo_math/items/efd44bc8f9229e014237
\usepackage{graphicx}
\usepackage{tikz}%https://www.mathcha.io/ が便利
\usetikzlibrary{shadows}
\usepackage{otf}%フォントメトリックがまともになる
\usepackage[T1]{fontenc}%上と同様
\usepackage{lmodern}%デフォルト欧文フォントを使う時だけ、いらんかも
\usepackage{comment}%複数行コメント
\usepackage{amsmath,amssymb,amsthm}%
\usepackage{amsfonts}%おなじみの数式パッケージたち
\usepackage{mathtools}%数式コマンドを追加
\usepackage{bbm,bm}%太字ベクトル
\usepackage{hyperref}%ハイパーリンクが貼れる
\usepackage{pxjahyper}%リハイパーンクを日本語に対応
\usepackage{physics}%偏微分とかが簡単に書ける
\usepackage{float}%画像とかの配置を強制できる
\usepackage{cancel}%打ち消し線が書ける
\usepackage{xcolor}
\usepackage{listings}%日本語入りのソースコードが書ける
\usepackage{../packages/jlisting}
\usepackage{tcolorbox}
\tcbuselibrary{breakable, skins, theorems}%定理環境
\usepackage{../packages/texex}%自作パッケージ2020あるいは2021 C:\texlive\2020\texmf-dist\tex\uplatex\texex

\begin{document}
\title{てんぷれ}
\author{062100873 髙柳海斗}
\date{\today}
\maketitle

\section{最初に}
前略

\section{最後に}
後略

\lstset{
	style=customlst,
}
\begin{lstlisting}[caption = sampe1 , label = code1 ,language=GLSL]
precision mediump float;
uniform float time;
uniform vec2  mouse;
uniform vec2  resolution;

void main(void){
	vec2 p = (gl_FragCoord.xy * 2.0 - resolution) / min(resolution.x, resolution.y);
	vec2 color = (vec2(1.0) + p.xy) * 0.5;
	gl_FragColor = vec4(color, 0.0, 1.0);
}
\end{lstlisting}

\end{document}
